\documentclass{procreport}

\usepackage[dvipdfm]{graphicx}
\usepackage{ascmac}
\usepackage{moreverb}

\title{プログラミングC第2回レポート課題}
\author{小山 亮}
\date{平成28年8月4日}
\担当教員{樽谷 優弥, まつ本 真佑}
\所属{計算機科学コース}
\学年{2年}
\学籍番号{09B15028}
\email{u745409b@ecs.osaka-u.ac.jp}

\begin{document}
\maketitle

\section{課題内容}
以下の機能を有するサブセット版のシェルをC言語で作成せよ。

\begin{itemize}
\item 外部コマンドの実行機能
\begin{itemize}
\item このプログラムに実装されていないコマンドに関して、このプログラムを実行しているシェルのコマンドを実行できるようにせよ。
\end{itemize}
\item ディレクトリの管理機能
\begin{itemize}
\item cdコマンド - カレントディレクトリを指定したディレクトリに、指定されていない場合はホームディレクトリに、移動できるようにせよ。
\item pushdコマンド - ディレクトリスタックの一番上にカレントディレクトリを保存できるようにせよ。
\item dirsコマンド - ディレクトリスタックの内容を表示できるようにせよ。
\item popdコマンド - ディレクトリスタックの一番上のディレクトリにカレントディレクトリを移動できるようにせよ。
\end{itemize}
\item ヒストリー機能
\begin{itemize}
\item historyコマンド - 実行したコマンドを保存しておき、そのリストを実行した順番とともに出力できるようにせよ。
\item !!コマンド - 前回実行したコマンドを実行できるようにせよ。
\item ![string]コマンド - 保存したコマンドのうち[string]に前方一致する最新のコマンドを実行できるようにせよ。
\end{itemize}
\item ワイルドカード機能
\begin{itemize}
\item ``*''と入力された部分をカレントディレクトリ内のすべてのディレクトリ、ファイルに適用されるようにせよ。
\end{itemize}
\item プロンプト機能
\begin{itemize}
\item プロンプトを指定した文字列に変更できるようにせよ。
\end{itemize}
\item スクリプト機能
\begin{itemize}
\item このプログラム実行時にリダイレクトで受け渡されたファイルからもコマンドを実行できるようにせよ。
\end{itemize}
\item エイリアス機能
\begin{itemize}
\item aliasコマンド - コマンドに対して指定した文字列が入力された時も実行できるようにせよ。
\item unaliasコマンド -コマンドに指定した文字列の関連付けを解除できるようにせよ。
\end{itemize}
\end{itemize}

またこれら以外にも1つ以上、自分で考えた機能を実装せよ。





\section{プログラム全体の説明}
C言語を用いてサブセット版のシェルを作成した。指定されたコマンドはプログラムの中で実行し、実装されていないコマンドについては外部コマンドとしてプログラムを実行しているシェルで実行されるようにした。

\subsection{シェルの仕様}
本章では実装した機能について説明する。
\begin{itemize}
\item 外部コマンドの実行機能
\begin{itemize}
\item このプログラムに実装されていないコマンドを外部コマンドとしてこのプログラムを実行しているシェルで実行されるようにした。
\end{itemize}
\item ディレクトリの管理機能
\begin{itemize}
\item cdコマンド - カレントディレクトリを指定したパスに移動できるようにした。
\item pushdコマンド - ディレクトリスタックを作成しスタックのの一番上に指定したパスをを保存できるようにした。指定しない場合はカレントディレクトリを保存する。
\item dirsコマンド - ディレクトリスタックの内容を表示できるようにした。
\item popdコマンド - ディレクトリスタックの一番上のディレクトリにカレントディレクトリを移動できるようにした。
\end{itemize}
\item ヒストリー機能
\begin{itemize}
\item historyコマンド - 実行したコマンドを構造体リストに保存しておき、そのコマンドのリストを実行した順番とともに出力できるようにした。保存できるコマンドは最新の32個までである。
\item !!コマンド - 前回実行したコマンドを実行できるようにした。
\item ![string]コマンド - 保存したコマンドのうち[string]に前方一致する最新のコマンドを実行できるようにした。
\end{itemize}
\item ワイルドカード機能
\begin{itemize}
\item ``*''と入力された部分をカレントディレクトリ内のすべてのディレクトリ、ファイルに適用されるようにした。
\end{itemize}
\item プロンプト機能
\begin{itemize}
\item プロンプトを指定した文字列に変更できるようにした。指定されない場合は``prompt''となる。
\end{itemize}
\item スクリプト機能
\begin{itemize}
\item このプログラム実行時にリダイレクトで指定されたファイルからもコマンドを実行できるようにした。
\end{itemize}
\item エイリアス機能
\begin{itemize}
\item aliasコマンド - コマンドを指定した文字列が入力された時も実行できるようにした。
\item unaliasコマンド -コマンドに指定した文字列の関連付けを解除できるようにした。
\end{itemize}
\end{itemize}

\subsection{処理の流れ・実装方法}
まず、コマンドラインを配列に読み込む。次に関数によってコマンドラインを引数として単語ごとに分ける。実際には、コマンドラインの単語の先頭アドレスをポインタ配列に代入していく。その後、ポインタ配列の1つ目にはコマンドが入っているのでif文によってコマンドごとに行う処理を変える。まず実装しているコマンドであればif文で分岐してそれぞれの機能が実現される。実装されていないコマンドであれば、まずaliasコマンドによって登録されたコマンドでないかを確認する。登録されたコマンドでなければ、ポインタ配列が関数に渡され、外部コマンドとして実行される。プログラム終了の際は、構造体リストや構造体二分木のために動的に確保した領域を開放する。



\section{外部コマンド実行機能}

\subsection{仕様}
このプログラムに実装されていないコマンドを外部コマンドとして実行する。コマンドラインの最後に``\&''があればバックグラウンドで実行する。

\subsection{処理の流れ}
子プロセスを生成し、その中でコマンドを実行する。コマンドの処理が終われば子プロセス上で実行しているプログラムを終了させる。バックグラウンドでの実行であれば、親プロセスは子プロセスの終了を待たずに次のプロンプトを表示し入力待ちとなる。フォアグラウンドでの実行であれば、親プロセスは子プロセスの終了を待つ。

\subsection{実装方法}
関数execvpによって実装した。関数の引数はコマンドラインを分割したもの、コマンドの状態である。

\subsection{テスト}
\subsubsection{テスト方法}
アプリケーションemacsを起動して、コマンドpsによって起動を確認する。
\subsubsection{テスト結果}
\begin{screen}
\begin{verbatim}
prompt : ps
  PID TT  STAT     TIME COMMAND
13233  1  Is+   0:00.04 -csh (csh)
13267  2  Ss    0:00.08 csh
13294  2  I    19:02.81 firefox
14987  2  S+    0:00.00 ./mysh
14990  2  R+    0:00.00 ps

prompt : emacs &

prompt : ps
  PID TT  STAT     TIME COMMAND
13233  1  Is+   0:00.04 -csh (csh)
13267  2  Is    0:00.08 csh
13294  2  I    19:04.46 firefox
14987  2  S+    0:00.00 ./mysh
15007  2  S+    0:00.32 emacs (emacs-24.5)
15012  2  R+    0:00.00 ps

\end{verbatim}
\end{screen}
コマンドによってemacsが起動したことがわかる。

\section{ディレクトリの管理機能}

\subsection{仕様}
この機能にはコマンドcd、pushd、dirs、popdがある。コマンドcdは指定したパスに移動するのみであるが、コマンドpushdは指定したパスに移動するだけでなく、ディレクトリスタックに保存する。コマンドdirsはディレクトリスタックにあるディレクトリのリストを表示する。コマンドpopdはディレクトリスタックの一番上にあるパスに移動する。

\subsection{処理の流れ}
まずif文の分岐によってそれぞれの処理に入る。cdはカレントディレクトリを移動する関数getcwdを用いた。pushdはメモリ容量を確保し、スタックの一番初めに追加する。dirsはスタックの内容を構造体リストを追って出力する。出力はプッシュした逆順である。popdはスタックの一番上にあるパスを出力する。

\subsection{実装方法}
cdは関数chdirによって実装した。
pushd、dirs、popdは構造体リストを用いてスタックを実現した。コマンドpushdが実行されるたびに、ディレクトリ
\subsection{テスト}
\subsubsection{テスト方法}
様々な引数(ディレクトリ)を指定して、数回ずつ実行してみる。
\subsubsection{テスト結果}
まずcdの実行結果について以下に示す。
\begin{screen}
\begin{verbatim}
prompt : cd /.amd_mnt/home/exp/exp5/r-koyama
/.amd_mnt/home/exp/exp5/r-koyama

prompt : cd r-koyama/pro-s2016
r-koyama/pro-s2016: No such file or directory
/.amd_mnt/home/exp/exp5/r-koyama

prompt : cd pro-c2016
/.amd_mnt/home/exp/exp5/r-koyama/pro-c2016

prompt : cd .
/.amd_mnt/home/exp/exp5/r-koyama/pro-c2016

prompt : cd ..
/.amd_mnt/home/exp/exp5/r-koyama

prompt : cd pro-c2016
/.amd_mnt/home/exp/exp5/r-koyama/pro-c2016

prompt : cd 2nd
/.amd_mnt/home/exp/exp5/r-koyama/pro-c2016/2nd

prompt : cd ../1st
/.amd_mnt/home/exp/exp5/r-koyama/pro-c2016/1st
\end{verbatim}
\end{screen}
絶対パスでも相対パスでもコマンドが実行されていることがわかる。また、存在しないディレクトリを指定するとエラーを返していることがわかる。次にpushd,dirs,popdの実行結果を以下に示す。
\begin{screen}
\begin{verbatim}
prompt : pushd
/.amd_mnt/home/exp/exp5/r-koyama/pro-c2016/1st

prompt : pushd ..
/.amd_mnt/home/exp/exp5/r-koyama/pro-c2016

prompt : pushd
/.amd_mnt/home/exp/exp5/r-koyama/pro-c2016

prompt : pushd 2nd
/.amd_mnt/home/exp/exp5/r-koyama/pro-c2016/2nd

prompt : pushd ../1st
/.amd_mnt/home/exp/exp5/r-koyama/pro-c2016/1st

prompt : dirs
/.amd_mnt/home/exp/exp5/r-koyama/pro-c2016/1st
/.amd_mnt/home/exp/exp5/r-koyama/pro-c2016/2nd
/.amd_mnt/home/exp/exp5/r-koyama/pro-c2016
/.amd_mnt/home/exp/exp5/r-koyama/pro-c2016
/.amd_mnt/home/exp/exp5/r-koyama/pro-c2016/1st

prompt : popd
/.amd_mnt/home/exp/exp5/r-koyama/pro-c2016/1st

prompt : popd
/.amd_mnt/home/exp/exp5/r-koyama/pro-c2016/2nd

prompt : popd
/.amd_mnt/home/exp/exp5/r-koyama/pro-c2016

prompt : popd
/.amd_mnt/home/exp/exp5/r-koyama/pro-c2016

prompt : popd
/.amd_mnt/home/exp/exp5/r-koyama/pro-c2016/1st
\end{verbatim}
\end{screen}
pushdに関して、ディレクトリを指定していない場合はカレントディレクトリをスタックにプッシュし、ディレクトリを指定した場合は、そのディレクトリに移動し、スタックにプッシュしていることがわかる。dirsに関して、ディレクトリスタックのディレクトリを上から新しい順に表示していることがわかる。popdに関してディレクトリスタックの一番上にあるディレクトリに移動していることがわかる。


\section{ヒストリー機能}

\subsection{仕様}
実行したコマンドを32個まで保存し、コマンドhistoryによってその内容を出力する。コマンド!!によって前回実行したコマンドを実行する。コマンド![string]によって[string]に前方一致する最新のコマンドを実行する。また[string]が数字であった場合10進数のint型に変換することによって、数字に準ずる履歴を実行することもできる。
\subsection{処理の流れ}
まずif文の分岐によってそれぞれの処理に入る。historyは配列の内容を下に一番新しいコマンドが来るように順に出力する。!!は最新のコマンドを実行する。![string]は[string]に前方一致するコマンドを新しい方から検索し、あれば実行する。
\subsection{実装方法}
配列を用意し、履歴を保存していく。historyは配列の中身を入っている分だけ出力する。!!は最新のコマンドをコマンドラインに移し、初めから処理を行う。![string]は文字か数字かを判断し文字であればそれに前方一致するコマンドを検索する。数字であれば、アスキーコードから数字を判断し一致する番号の履歴を探す。
\subsection{テスト}
\subsubsection{テスト方法}
他の機能のテストを行った後にこれらのコマンドを実行してみて動作を確認する。
\subsubsection{テスト結果}
\begin{screen}
\begin{verbatim}
prompt : history
 1 pwd
 2 ps
 3 ls -a
 4 history
 5 pushd ..
 6 pushd 1st
 7 ls
 8 emacs &
 9 ps
10 kill 14141
11 emacs groupA
12 history

prompt : !1
pwd
/.amd_mnt/home/exp/exp5/r-koyama/pro-c2016/1st

prompt : !9
ps
  PID TT  STAT     TIME COMMAND
13233  1  Is+   0:00.04 -csh (csh)
13267  2  Is    0:00.05 csh
13294  2  I    16:08.68 firefox
14137  2  S+    0:00.01 ./mysh
14141  2  Z+    0:00.43 <defunct>
14153  2  R+    0:00.00 ps

prompt : !p
ps
  PID TT  STAT     TIME COMMAND
13233  1  Is+   0:00.04 -csh (csh)
13267  2  Is    0:00.05 csh
13294  2  I    16:09.00 firefox
14137  2  S+    0:00.01 ./mysh
14141  2  Z+    0:00.43 <defunct>
14154  2  R+    0:00.00 ps

prompt : !!
ps
  PID TT  STAT     TIME COMMAND
13233  1  Is+   0:00.04 -csh (csh)
13267  2  Is    0:00.05 csh
13294  2  I    16:09.00 firefox
14137  2  S+    0:00.01 ./mysh
14141  2  Z+    0:00.43 <defunct>
14154  2  R+    0:00.00 ps

\end{verbatim}
\end{screen}
historyコマンドによってコマンドの履歴が降順に表示されていることがわかる。また、![string]や![num]も実装されていることがわかる。!!コマンドも実現されている。

\section{ワイルドカード機能}

\subsection{仕様}
コマンドラインに``*''があれば、カレントディレクトリのディレクトリとファイル全てに置き換える。
\subsection{処理の流れ}
引数のポインタ配列の``*''の場所をカレントディレクトリのディレクトリとファイル全てに置き換える。
\subsection{実装方法}
コマンドラインを分割する段階で、``*''があれば、カレントディレクトリのディレクトリとファイル全てにスペースで区切って置き換える。
\subsection{テスト}
\subsubsection{テスト方法}
このプログラムには``*''を使用できるコマンドがないので外部コマンドを用いてテストする。
\subsubsection{テスト結果}
\begin{screen}
\begin{verbatim}
prompt : ls
#simple_shell.c#        mysh.c~                 simple_shell-original
mysh                    prog.txt                simple_shell-original.c
mysh.c                  simple_shell            simple_shell.c~

prompt : mv * ../2ndtemp2
mv: rename . to ../2ndtemp2/.: Invalid argument
mv: rename .. to ../2ndtemp2/..: Invalid argument

prompt : cd ../2ndtemp2
/.amd_mnt/home/exp/exp5/r-koyama/pro-c2016/2ndtemp2

prompt : ls
#simple_shell.c#        mysh.c~                 simple_shell-original
mysh                    prog.txt                simple_shell-original.c
mysh.c                  simple_shell            simple_shell.c~

prompt : rm *
rm: "." and ".." may not be removed

prompt : ls

prompt : exit
done.
\end{verbatim}
\end{screen}
ワイルドカード機能は実装されているとわかる。

\section{プロンプト機能}

\subsection{仕様}
指定した文字列にプロンプトを変更する。
\subsection{処理の流れ}
プロンプトの変数を用意し、常にその中にプロンプトの文字列を保有する。コマンドの入力待ちのときにプロンプトとしてこの文字列を表示する。
\subsection{実装方法}
``prompt''で初期化されたchar型配列を用意し、このコマンドによって、第一引数の文字列に変更する。引数がない場合は、``prompt''とする。
\subsection{テスト}
\subsubsection{テスト方法}
文字列を指定して数回この機能を試してみる。
\subsubsection{テスト結果}
\begin{screen}
\begin{verbatim}
prompt : prompt ryo

ryo : prompt command

command : prompt

prompt :
\end{verbatim}
\end{screen}
引数に指定した文字列にプロンプトを変更していることがわかる。また、引数を指定しない場合は、``prompt''と変更されていることがわかる。

\section{スクリプト機能}

\subsection{仕様}
このプログラム実行時にリダイレクトで受けとったファイルから一行ずつ文字列を読み取り、コマンドとして実行する。
\subsection{処理の流れ}
関数fgetsで一行を読み取りコマンドラインとしてバッファに収める。その後はキーボード入力の場合と同じである。
\subsection{実装方法}
関数isattyによって入力リダイレクトがあるかを判定し、あればプロンプトを表示しない。
\subsection{テスト}
\subsubsection{テスト方法}
テキストデータを示し、それを使用した結果を示す。
\subsubsection{テスト結果}
\label{sec:script}
使用したテキストファイル
\begin{screen}
\begin{verbatim}
prog.txt

ls
pushd ..
ls
dirs
popd
pwd
\end{verbatim}
\end{screen}
テスト結果
\begin{screen}
\begin{verbatim}
exp175[81]% ./simple_shell < prog.txt
#simple_shell.c#        simple_shell            simple_shell.c
prog.txt                simple_shell-original   simple_shell.c~
prog.txt~               simple_shell-original.c
/.amd_mnt/home/exp/exp5/r-koyama/pro-c2016
11                              sample-procreport.tex
12                              sample10-1
13                              sample10-1.c
1st                             sample10-1.c~
2nd                             sample10-2
a.out                           sample10-2.c
counter.c                       sample10-2.c~
counter.c~                      sample10-3
counter.exe                     sample10-3.c
data.txt                        sample8-3.c
ffproxy-1.6                     sample8-3.c~
ffproxy-1.6.tar.gz              sample8-3.exe
foo                             sample9-2
hoge.txt                        sample9-2_main.c
mori.txt                        sample9-2_main.c~
perl                            sample9-2_main.o
proC-3                          sample9-2_product.c
report1table.pl                 sample9-2_product.c~
sample-2-1.aux                  sample9-2_product.h
sample-2-1.dvi                  sample9-2_product.h~
sample-2-1.log                  sample9-2_product.o
sample-2-1.pdf                  unix_commands_private.txt
sample-2-1.tex                  unix_commands_shared.txt
sample-2-1.tex~                 unix_commands_shared.txt~
sample-procreport.log
/.amd_mnt/home/exp/exp5/r-koyama/pro-c2016
/.amd_mnt/home/exp/exp5/r-koyama/pro-c2016
/.amd_mnt/home/exp/exp5/r-koyama/pro-c2016
\end{verbatim}
\end{screen}

外部コマンドも内部コマンドも実行されていることがわかる。またexitコマンドがなくても正常に終了していることがわかる。

\section{エイリアス機能}

\subsection{仕様}
aliasコマンドでコマンドを指定した文字列を入力しても実行されるようになる。unaliasコマンドでaliasコマンドで指定した文字列を解除する。
\subsection{処理の流れ}
aliasコマンドは構造体の領域を動的に確保し、構造体に第一引数の文字列と第二引数のコマンドを保存する。unaliasコマンドは第一引数で指定された文字列の構造体を探し、その構造体を削除する。
\subsection{実装方法}
構造体二分木を使用した。指定した文字列の辞書順になっている。削除の場合は指定した文字列を探し、その右側から一番小さいものを探し出し削除したいものと入れ替えることで削除を実現している。
\subsection{テスト}
\subsubsection{テスト方法}
いくつかのコマンドに対してこの機能を試してみる。
\subsubsection{テスト結果}
\begin{screen}
\begin{verbatim}
prompt : alias cdir pwd

prompt : alias h history

prompt : cdir
/.amd_mnt/home/exp/exp5/r-koyama/pro-c2016/2nd

prompt : h
 1 alias cc pwd
 2 alias h history
 3 pwd
 4 history

prompt : unalias h

prompt : h

prompt : unalias cdir

prompt : cdir
\end{verbatim}
\end{screen}

aliasコマンドによって、文字列を第二引数のコマンドに対して設定していることがわかる。そのコマンドを打つと実際に置き換えられていることがわかる。unaliasコマンドによって文字列の指定を解除していることがわかる。

\section{自分で考えた機能}
自分で考えた機能としてcatコマンドを実装した。
\subsection{仕様}
第一引数で指定されたファイルの中身を出力する。
\subsection{処理の流れ}
第一引数で指定されたファイルを開く。その中身を一行ずつ読みとり出力する。ファイルを閉じる。
\subsection{実装方法}
ファイルポインタを宣言し、ファイルを操作することによって実装した。
\subsection{テスト}
\subsubsection{テスト方法}
\ref{sec:script}章で用いたテキストファイルに対してこの機能を試してみる。
\subsubsection{テスト結果}
\begin{screen}
\begin{verbatim}
prompt : cat prog.txt
ls
pushd ..
ls
dirs
popd
pwd
\end{verbatim}
\end{screen}
ファイルの中身が表示されていることがわかる。

\section{その他実装した機能}
ディレクトリの管理機能のデバッグを簡単にするためにコマンドpwdを実装した。カレントディレクトリを取得し、それを出力するという方法で実装を実現した。


\section{工夫点}
\begin{itemize}
\item history機能で![num]コマンドも実装した。
\item aliasコマンドのデータ構造として、二分木を使用したため、検索にかかる時間を抑えた。
\item pushdコマンドでカレントディレクトリだけでなくパスを指定してスタックに保存して移動できるようにした。
\end{itemize}

\section{考察}
\label{sec:kousatsu}
このプログラムは既存のシェルに機能の完成度が劣っている。既存のシェルにはそれぞれのコマンドにオプションを付けられる。このプログラムにも場合分けなどを使用してオプションを実装する余地がある。これからそのような機能を実装したいと思う。またプロンプト機能には現在の時刻やカレントディレクトリを表示させたりすることも改造しだいでは可能である。


\section{感想}
始めは自分でシェルを作ることに全く意味を見いだせなかったが、機能を実装していくうちに自分のシェルをもっとよくしていきたいという気持ちがわいてきた。\ref{sec:kousatsu}章で述べた機能などを実装して、今のプログラムを機能がより充実したものにしたいと思う。また、子プロセス生成などの新しいコマンドは始めは理解に苦しんだが、触れることができてよかった。



\appendix
\newpage

\section{プログラムリスト}
\begin{verbatimtab}
     1	/*
     2	 *  インクルードファイル
     3	 */
     4	
     5	#include <stdio.h>
     6	#include <stdlib.h>
     7	#include <sys/types.h>
     8	#include <unistd.h>
     9	#include <sys/wait.h>
    10	#include <string.h>
    11	#include <dirent.h>
    12	
    13	/*
    14	 *  定数の定義
    15	 */
    16	
    17	#define BUFLEN	1024	 /* コマンド用のバッファの大きさ */
    18	#define MAXARGNUM  256	 /* 最大の引数の数 */
    19	
    20	#define MAXHISTORY 32
    21	#define PATHLEN 256
    22	#define PROMPTLEN 32
    23	#define COMMANDLEN 16
    24	#define DIRLEN 64
    25	#define WILDCARDNUM 64
    26	
    27	/*
    28	 *	構造体の宣言
    29	 */
    30	
    31	typedef struct dirs_stack{
    32		char dirsname[PATHLEN];
    33		struct dirs_stack *next;
    34	} DIRS_STACK;
    35		
    36	typedef struct alias_binary{
    37		char alias_command[COMMANDLEN];
    38		char original_command[BUFLEN];
    39		struct alias_binary *left;
    40		struct alias_binary *right;
    41	} ALIAS_BINARY;
    42	
    43	/*
    44	 *  ローカルプロトタイプ宣言
    45	 */
    46	
    47	int parse(char [], char *[]);
    48	void execute_command(char *[], int);
    49	
    50	int execute_alias(ALIAS_BINARY *, char *, char *);
    51	void delete_alias(ALIAS_BINARY **, char *);
    52	void show_alias(ALIAS_BINARY *);
    53	void delete_binary(ALIAS_BINARY *);
    54	
    55	/*----------------------------------------------------------------------------
    56	 *
    57	 *  関数名   : main
    58	 *
    59	 *  作業内容 : シェルのプロンプトを実現する
    60	 *
    61	 *  引数	 :
    62	 *
    63	 *  返り値   :
    64	 *
    65	 *  注意	 :
    66	 *
    67	 *--------------------------------------------------------------------------*/
    68	
    69	int main(int argc, char *argv[])
    70	{
    71		char command_buffer[BUFLEN]; /* コマンド用のバッファ */
    72		char *args[MAXARGNUM];	   /* 引数へのポインタの配列 */
    73		int command_status;		  /* コマンドの状態を表す
    74										
command_status = 0 : フォアグラウンドで実行
    75										
command_status = 1 : バックグラウンドで実行
    76										
command_status = 2 : シェルの終了
    77										
command_status = 3 : 何もしない */
    78		char history[MAXHISTORY][BUFLEN];
    79		int history_num[MAXHISTORY] = {};
    80		int num = 0;
    81		int prev =0;
    82		char pathname[PATHLEN];
    83		char *command_buffer_p;
    84		char prompt[PROMPTLEN] = "prompt";
    85		int i;
    86		int history_x;
    87		
    88		DIRS_STACK *head = NULL, *p, *access;
    89		ALIAS_BINARY *a_head = NULL, *a_p, *a_access, *a_temp;
    90		/*
    91		 *  無限にループする
    92		 */
    93	
    94		for(;;) {
    95			/*
    96			 *  プロンプトを表示する
    97			 */
    98	
    99			if(prev == 0){
   100				if(isatty(fileno(stdin)))
   101					printf("\n%s : ", prompt);
   102	
   103			/*
   104			 *  標準入力から1行を command_buffer へ読み込む
   105			 *  入力が何もなければ改行を出力してプロンプト表示へ戻る
   106			 */
   107				if(fgets(command_buffer, BUFLEN, stdin) == NULL) {
   108	
   109				  if(isatty(fileno(stdin))){
   110				    printf("\n");
   111					continue;
   112				  }else{
   113				    exit(EXIT_SUCCESS);
   114				  }
   115				}
   116				if(command_buffer[strlen(command_buffer) - 1] 
== '\n'){
   117					command_buffer[strlen(command_buffer) - 1] 
= '\0';
   118				}
   119			}
   120			
   121			command_buffer_p = &command_buffer[0];
   122			
   123			while(*command_buffer_p == ' ' || *command_buffer_p == '\t') {
   124				command_buffer_p++;
   125			}
   126		
   127			if(strcmp(command_buffer_p, "") != 0){
   128				if(prev == 0)
   129					for(i = 0; i<MAXHISTORY-1; i++){
   130						strcpy(&history[i][0], 
&history[i+1][0]);
   131						history_num[i] = history_num[i+1];
   132					}
   133				strcpy(&history[MAXHISTORY-1][0], command_buffer);
   134				history_num[MAXHISTORY-1] = ++num;
   135			}
   136			
   137			prev = 0;
   138	
   139			/*
   140			 *  入力されたバッファ内のコマンドを解析する
   141			 *
   142			 *  返り値はコマンドの状態
   143			 */
   144	
   145			command_status = parse(command_buffer, args);
   146	
   147			/*
   148			 *  終了コマンドならばプログラムを終了
   149			 *  引数が何もなければプロンプト表示へ戻る
   150			 */
   151	
   152			if(command_status == 2) {
   153				access = head;					//スタック容量を開放
   154				while(access != NULL){
   155					access = access->next;
   156					free(head);
   157					head = access;
   158				}
   159				delete_binary(a_head);			//構造体二分木容量開放
   160				printf("done.\n");
   161				exit(EXIT_SUCCESS);
   162			} else if(command_status == 3) {
   163				continue;
   164			}
   165	
   166			/*
   167			 *  コマンドを実行する
   168			 */
   169	
   170			if(!strcmp(args[0], "history")){
   171				for(i = 0; i<MAXHISTORY; i++)
   172					if(history_num[i] != 0)
   173						printf("%2d %s\n",history_num[i], 
&history[i][0]);
   174					
   175			}else if(!strncmp(args[0], "!", 1)){
   176				++args[0];
   177				if(!strcmp(args[0], "!")){
   178					printf("%s\n",&history[MAXHISTORY-2][0]);
   179					stpcpy(&command_buffer[0], 
&history[MAXHISTORY-2][0]);
   180					--num;
   181					prev = 1;
   182				}else{
   183					history_x = 0;
   184					for(i = 0; *(args[0]+i) != '\0'; i++)
   185						if(47 < *(args[0]+i) && 
*(args[0]+i) < 58){
   186							history_x *= 10;
   187							history_x += 
(int)*(args[0]+i) - 48;
   188						}
   189						else
   190							break;
   191					if(*(args[0]+i) == '\0'){
   192						for(i = 0; i<MAXHISTORY && 
prev == 0; i++)
   193							if(history_x == 
history_num[MAXHISTORY-1-i]){
   194								printf(
"%s\n",&history[MAXHISTORY-1-i][0]);
   195								strcpy(
command_buffer, &history[MAXHISTORY-1-i][0]);
   196								--num;
   197								prev = 1;
   198							}
   199						if(prev == 0)
   200							printf("%s: event not found\n", args[0]-1);
   201					}else{
   202						for(i = 0; i<MAXHISTORY && 
prev == 0; i++)
   203							if(!strncmp(args[0], 
&history[MAXHISTORY-1-i][0], strlen(args[0]))){
   204								printf("%s\n",
&history[MAXHISTORY-1-i][0]);
   205								strcpy(
command_buffer, &history[MAXHISTORY-1-i][0]);
   206								--num;
   207								prev = 1;
   208							}
   209						if(prev == 0)
   210							printf(
"%s: event not found\n", args[0]-1);
   211					}
   212				}
   213				
   214			}else if(!strcmp(args[0], "cd")){
   215				if(chdir(args[1]))
   216					printf("%s: No such file or directory\n", 
args[1]);
   217				getcwd(pathname, PATHLEN);
   218				printf("%s\n", pathname);
   219				
   220			}else if(!strcmp(args[0], "pwd")){
   221				getcwd(pathname, PATHLEN);
   222				printf("%s\n", pathname);
   223				
   224			}else if(!strcmp(args[0], "pushd")){
   225				p = (DIRS_STACK*)malloc(sizeof(DIRS_STACK));
   226				if(args[1] == NULL){
   227					getcwd(p->dirsname, PATHLEN);
   228					printf("%s\n",p->dirsname);
   229					p->next = head;
   230					head = p;
   231				}else{
   232					if(chdir(args[1])){
   233						printf(
"%s: No such file or directory\n", args[1]);
   234						free(p);
   235					}else{
   236						getcwd(p->dirsname, PATHLEN);
   237						printf("%s\n",p->dirsname);
   238						p->next = head;
   239						head = p;
   240					}
   241				}
   242				
   243			}else if(!strcmp(args[0], "dirs")){
   244				if(head == NULL)
   245					printf("(null)\n");
   246				else{
   247					access = head;
   248					while(access != NULL){
   249						printf("%s\n",access->dirsname);
   250						access = access->next;
   251					}
   252				}
   253	
   254			}else if(!strcmp(args[0], "popd")){
   255				if(head == NULL)
   256					printf("(null)\n");
   257				else{
   258					chdir(head->dirsname);
   259					printf("%s\n",head->dirsname);
   260					p = head;
   261					head = p->next;
   262					free(p);
   263				}
   264	
   265			}else if(!strcmp(args[0], "prompt")){
   266				if(args[1] == NULL)
   267					strcpy(prompt, "prompt");
   268				else
   269					stpcpy(prompt, args[1]);
   270				
   271			}else if(!strcmp(args[0], "alias")){
   272				if(args[1] == NULL){
   273					show_alias(a_head);
   274				}
   275				else{
   276					a_access = a_head;
   277					while(a_access != NULL){
   278						a_temp = a_access;
   279						if(strcmp(a_access->alias_command, 
args[1]) > 0)
   280							a_access = a_access->left;
   281						else if(strcmp(a_access->alias_command, 
args[1]) < 0)
   282							a_access = a_access->right;
   283						else{
   284							strcpy(
a_access->original_command, args[2]);
   285							i = 3;
   286							while(args[i] != NULL){
   287								strcat(
a_access->original_command, " ");
   288								strcat(
a_access->original_command, args[i]);
   289								++i;
   290							}
   291							break;
   292						}
   293					}
   294					if(a_access == NULL){
   295						a_p = 
(ALIAS_BINARY *)malloc(sizeof(ALIAS_BINARY));
   296						strcpy(a_p->alias_command, args[1]);
   297						strcpy(a_p->original_command, args[2]);
   298						i = 3;
   299						while(args[i] != NULL){
   300							strcat(
a_access->original_command, " ");
   301							strcat(
a_access->original_command, args[i]);
   302							++i;
   303						}
   304						a_p->left = NULL;
   305						a_p->right = NULL;
   306						if(a_head == NULL)
   307							a_head = a_p;
   308						else{
   309							if(strcmp(
a_temp->alias_command, args[1]) > 0)
   310								a_temp->left = a_p;
   311							else if(strcmp(
a_temp->alias_command, args[1]) < 0)
   312								a_temp->right = a_p;
   313						}
   314					}
   315				}
   316			}else if(!strcmp(args[0], "unalias")){
   317				delete_alias(&a_head, args[1]);
   318			}else{
   319				if(execute_alias(a_head, args[0], command_buffer)){
   320					--num;
   321					prev = 1;
   322				}else
   323					execute_command(args, command_status);
   324			}
   325		}
   326	
   327		return 0;
   328	}
   329	
   330	int execute_alias(ALIAS_BINARY *now, char *command, char *command_buffer)
   331	{
   332		ALIAS_BINARY *access;
   333		
   334		access = now;
   335		while(access != NULL){
   336			if(strcmp(access->alias_command, command) > 0){
   337				access = access->left;
   338			}else if(strcmp(access->alias_command, command) < 0){
   339				access = access->right;
   340			}else{
   341				strcpy(command_buffer, access->original_command);
   342				return(1);
   343			}
   344		}
   345			return(0);
   346	}
   347	
   348	void delete_alias(ALIAS_BINARY **head, char *command)
   349	{
   350		ALIAS_BINARY *access1, *access2, *temp1, *temp2 = NULL;
   351		
   352		access1 = *head;
   353		
   354		while(access1 != NULL){
   355			if(strcmp(access1->alias_command, command) > 0){
   356				temp1 = access1;
   357				access1 = access1->left;
   358			}
   359			else if(strcmp(access1->alias_command, command) < 0){
   360				temp1 = access1;
   361				access1 = access1->right;
   362			}
   363			else
   364				break;
   365		}
   366		
   367		if(access1 == NULL)
   368			return;
   369		
   370		if(access1->right == NULL){
   371			if(access1 == *head)
   372				*head = access1->left;
   373			else{
   374				if(strcmp(temp1->alias_command, command) > 0)
   375					temp1->left = access1->left;
   376				else if (strcmp(temp1->alias_command, command) < 0)
   377					temp1->right = access1->left;
   378			}
   379			free(access1);
   380		}else{
   381			access2 = access1->right;
   382			while(access2->left != NULL){
   383				temp2 = access2;
   384				access2 = access2->left;
   385			}
   386			if(temp2 != NULL)
   387				temp2->left = access2->right;
   388			access2->left = access1->left;
   389			access2->right = access1->right;
   390			if(access1 == *head)
   391				*head = access2;
   392			else{
   393				if(strcmp(temp1->alias_command, command) > 0)
   394					temp1->left = access2;
   395				else if (strcmp(temp1->alias_command, command) < 0)
   396					temp1->right = access2;
   397			}
   398			free(access1);
   399		}
   400	}
   401	
   402	void show_alias(ALIAS_BINARY *now)
   403	{
   404		if(now->left != NULL)
   405			show_alias(now->left);
   406		printf("%s: %s\n",now->alias_command,now->original_command);
   407		if(now->right != NULL)
   408			show_alias(now->right);
   409	}
   410	
   411	void delete_binary(ALIAS_BINARY *now)
   412	{
   413		if(now != NULL){
   414			delete_binary(now->left);
   415			delete_binary(now->right);
   416			free(now);
   417		}
   418	}
   419	
   420	/*----------------------------------------------------------------------------
   421	 *
   422	 *  関数名   : parse
   423	 *
   424	 *  作業内容 : バッファ内のコマンドと引数を解析する
   425	 *
   426	 *  引数	 :
   427	 *
   428	 *  返り値   : コマンドの状態を表す :
   429	 *				0 : フォアグラウンドで実行
   430	 *				1 : バックグラウンドで実行
   431	 *				2 : シェルの終了
   432	 *				3 : 何もしない
   433	 *
   434	 *  注意	 :
   435	 *
   436	 *--------------------------------------------------------------------------*/
   437	
   438	int parse(char buffer[],		/* バッファ */
   439			  char *args[])			/* 引数へのポインタ配列 */
   440	{
   441		int arg_index;   /* 引数用のインデックス */
   442		int status;   /* コマンドの状態を表す */
   443	
   444		char *args_temp[COMMANDLEN];
   445		char wild_card[WILDCARDNUM][DIRLEN];
   446		
   447		DIR *dir;
   448		struct dirent *dp;
   449		/*
   450		 *  変数の初期化
   451		 */
   452	
   453		arg_index = 0;
   454		status = 0;
   455	
   456		/*
   457		 *  バッファが終了を表すコマンド("exit")ならば
   458		 *  コマンドの状態を表す返り値を 2 に設定してリターンする
   459		 */
   460	
   461		if(strcmp(buffer, "exit") == 0) {
   462			status = 2;
   463			return status;
   464		}
   465		
   466		/*
   467		 *  バッファ内の文字がなくなるまで繰り返す
   468		 *  (ヌル文字が出てくるまで繰り返す)
   469		 */
   470	
   471		while(*buffer != '\0') {
   472	
   473			/*
   474			 *  空白類(空白とタブ)をヌル文字に置き換える
   475			 *  これによってバッファ内の各引数が分割される
   476			 */
   477	
   478			while(*buffer == ' ' || *buffer == '\t') {
   479				*(buffer++) = '\0';
   480			}
   481	
   482			/*
   483			 * 空白の後が終端文字であればループを抜ける
   484			 */
   485	
   486			if(*buffer == '\0') {
   487				break;
   488			}
   489	
   490			/*
   491			 *  空白部分は読み飛ばされたはず
   492			 *  buffer は現在は arg_index + 1 個めの引数の先頭を指している
   493			 *
   494			 *  引数の先頭へのポインタを引数へのポインタ配列に格納する
   495			 */
   496	
   497			args[arg_index] = buffer;
   498			++arg_index;
   499	
   500			/*
   501			 *  引数部分を読み飛ばす
   502			 *  (ヌル文字でも空白類でもない場合に読み進める)
   503			 */
   504	
   505			while((*buffer != '\0') && (*buffer != ' ') && 
(*buffer != '\t')) {
   506				++buffer;
   507			}
   508		}
   509	
   510		/*
   511		 *  最後の引数の次にはヌルへのポインタを格納する
   512		 */
   513	
   514		args[arg_index] = NULL;
   515	
   516		/*
   517		 *  最後の引数をチェックして "&" ならば
   518		 *
   519		 *  "&" を引数から削る
   520		 *  コマンドの状態を表す status に 1 を設定する
   521		 *
   522		 *  そうでなければ status に 0 を設定する
   523		 */
   524	
   525		if(arg_index > 0 && strcmp(args[arg_index - 1], "&") == 0) {
   526			--arg_index;
   527			args[arg_index] = '\0';
   528			status = 1;
   529		} else {
   530			status = 0;
   531		}
   532	
   533		/*
   534		 *  引数が何もなかった場合
   535		 */
   536	
   537		if(arg_index == 0) {
   538			status = 3;
   539		}
   540	
   541		if(arg_index > 0){				//ワイルドカード機能
   542	
   543			int i=1;
   544			while(args[i] != NULL && strcmp(args[i], "*") != 0){
   545			i++;
   546			}
   547			if(args[i] != NULL){
   548				int j=0;
   549				while(args[i+j] != NULL){
   550					args_temp[j] = args[i+1+j];
   551					j++;
   552				}
   553				j = 0;
   554				arg_index = i;
   555				dir = opendir(".");
   556				while ((dp = readdir(dir)) != NULL) {
   557					strcpy(&wild_card[j][0], dp->d_name);
   558					args[arg_index] = &wild_card[j][0];
   559					arg_index++;
   560					j++;
   561				}
   562				j = 0;
   563				while(args_temp[j] != NULL){
   564					args[arg_index] = args_temp[j];
   565					arg_index++;
   566					j++;
   567				}
   568				args[arg_index] = NULL;
   569				closedir(dir);
   570			}
   571		}
   572		
   573		/*
   574		 *  コマンドの状態を返す
   575		 */
   576	
   577		return status;
   578	}
   579	
   580	/*----------------------------------------------------------------------------
   581	 *
   582	 *  関数名   : execute_command
   583	 *
   584	 *  作業内容 : 引数として与えられたコマンドを実行する
   585	 *			 コマンドの状態がフォアグラウンドならば、コマンドを
   586	 *			 実行している子プロセスの終了を待つ
   587	 *			 バックグラウンドならば子プロセスの終了を待たずに
   588	 *			 main 関数に返る(プロンプト表示に戻る)
   589	 *
   590	 *  引数	 :
   591	 *
   592	 *  返り値   :
   593	 *
   594	 *  注意	 :
   595	 *
   596	 *--------------------------------------------------------------------------*/
   597	
   598	void execute_command(char *args[],	/* 引数の配列 */
   599						 int command_status)	 /* コマンドの状態 */
   600	{
   601		int pid;	  /* プロセスID */
   602		int status;   /* 子プロセスの終了ステータス */
   603		int wait;
   604		
   605		/*
   606		 *  子プロセスを生成する
   607		 *
   608		 *  生成できたかを確認し、失敗ならばプログラムを終了する
   609		 */
   610	
   611		/******** Your Program ********/
   612		pid = fork();
   613		if (pid < 0){
   614			printf("fork failure\n");
   615			exit(EXIT_FAILURE);
   616		}
   617	
   618		/*
   619		 *  子プロセスの場合には引数として与えられたものを実行する
   620		 *
   621		 *  引数の配列は以下を仮定している
   622		 *  ・第1引数には実行されるプログラムを示す文字列が格納されている
   623		 *  ・引数の配列はヌルポインタで終了している
   624		 */
   625	
   626		/******** Your Program ********/
   627		else if(pid == 0){
   628			execvp(args[0],args);
   629			exit(EXIT_SUCCESS); 
   630		}
   631	
   632		/*
   633		 *  コマンドの状態がバックグラウンドなら関数を抜ける
   634		 */
   635	
   636		/******** Your Program ********/
   637		else{
   638			if(command_status == 1)
   639				return;
   640	
   641		/*
   642		 *  ここにくるのはコマンドの状態がフォアグラウンドの場合
   643		 *
   644		 *  親プロセスの場合に子プロセスの終了を待つ
   645		 */
   646	
   647		/******** Your Program ********/
   648			int wait = waitpid(pid, &status, 0); //子プロセスのプロセスIDを指定して、終了を待つ
   649			if(wait < 0){
   650				printf("wait failure\n");
   651				exit(EXIT_FAILURE);
   652			}
   653		}
   654		
   655		return;
   656	}
   657	/*-- END OF FILE -----------------------------------------------------------*/
\end{verbatimtab}

\end{document}